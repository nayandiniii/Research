

\documentclass[openacc]{rsproca_new}%%%%where rsproca is the template name

%%%% *** Do not use packages/style files that adjust lengths, control margins, column widths, etc. ***

%%%%%%%%%%% Defining Enunciations  %%%%%%%%%%%
\newtheorem{theorem}{\bf Theorem}[section]
\newtheorem{condition}{\bf Condition}[section]
\newtheorem{corollary}{\bf Corollary}[section]
%%%%%%%%%%%%%%%%%%%%%%%%%%%%%%%%%%%%%%%%%%%%%%%

%%%%% Please insert respective article type here %%%%
\titlehead{Research}

\begin{document}

%%%% Article title to be placed here
\title{Node Generated Model Working for Social Media Handling using Federated Learning With Big Data}

\author{%%%% Author details
Nandini Sharma,\\2247153$^{1}$}

%%%%%%%%% Insert author address here
\address{$^{1}51,CL Layout,Hosur Main Road,Bengaluru$}


%%%% Subject entries to be placed here %%%%
\subject{Big Data, Federated Learning, Social Media & Approach}

%%%% Keyword entries to be placed here %%%%
\keywords{Privacy Concerns, Data Analytics, Social,Security,Federated Concepts }

%%%% Insert corresponding author and its email address}
\email{nandini.nandini@mca.christuniversity.in}

%%%% Abstract text to be placed here %%%%%%%%%%%%
\begin{abstract}
People all over the world are connected by social networking services (SNSs), where they share content, photos, videos, and comments, as well as follow their friends. The five Vs of big data—velocity, volume, value, variety, and veracity—make social networks unique.The analysis of social data to describe and discover communication patterns among users and comprehend their behaviors has received a lot of attention as a result of the ever-increasing growth of social networks.Federated learning (FL) is a type of machine learning system in which multiple clients collaborate with a central aggregator to solve problems. In order to safeguard the privacy of each device, this setting permits the dispersion of the training data.For applications like next-word prediction, face detection, voice recognition, and so on, federated learning can be used to build models of user behavior from a pool of social account data without disclosing any personal information. Google uses federated learning, for instance, to enhance on-device machine learning models like "Hey Google" in the Google Assistant, which enables users to issue voice commands.
This paper provides an overview of federated learning systems, with a focus on Social Media handling in today's society by analyzing the data collected from the users.This paper examines recent developments and provides a comprehensive list of unresolved issues, inspired by the rapid growth of FL research.The effects of Big Data on social media and other emerging technologies, as well as their eventual impact on society, will be the subject of this paper. It will discuss emerging and concurrent issues in education, the economy, advanced technology, the environment, and safety, as well as the impact of Big Data and Social Media on all of these areas.
\end{abstract}
%%%%%%%%%%%%%%%%%%%%%%%%%%%

\rsbreak

%%%%%%%%%% Insert the texts which can accomdate on firstpage in the tag "fmtext" %%%%%

\section{Isuue or Challenges}
%%%% Insert A head here

In today's fast-paced, ever-evolving world, where we are all connected to each other and free to share information, photos, and videos over a network, there are certain security and privacy concerns that need to be addressed. To address these issues, we are providing a solution by developing a model to protect user privacy using a model for node and analyzing the large amounts of data that have been collected.

\subsection{Security Issues}
%%%% Insert B head here
In recent years, a new field of data mining research known as privacy-concern data mining (PCDM) has received a lot of attention.
The fundamental idea behind PCDM is to modify the data in such a way that using data mining algorithms, the modification of the data should not compromise the security of sensitive information in the data.
Because many unintended disclosures of sensitive information may occur as a result of data mining, it is difficult for researchers to mitigate the privacy risk. Data provider, data collector, data miner, and decision maker are the four types of users involved in data mining applications that should be taken into consideration for security reasons.

\subsection{Management Challenges}
%%%% Insert B head here
Because people use social networking sites to connect with a wide range of people and share personal information, social media warehouses contain sensitive data, such as personal information. The access to such data raises ethical and legal issues. Therefore, the data must be secured, access controlled, and logged for audit purposes. Michael Blaha) Data security and privacy are the primary management obstacles..



%%%%%%%%%%%%%%% End of first page %%%%%%%%%%%%%%%%%%%%%

\maketitle


\section{Introduction}
\subsection{Big Data & Social Media}
Businesses thrive when they have a thorough understanding of their customers. As a result, keeping an eye on what people do online is becoming increasingly important to their success. As a crucial component for monitoring social media activity, particularly on social networking websites like Facebook, Twitter, and LinkedIn, businesses are investing in big data analytics.
Social media analytics are a compilation of internet users' actions. Organizations are able to gain timely and comprehensive insights into customers thanks to the availability of data on consumers' web browsing, online shopping habits, customer feedback, and marketing research conducted on social networks. As a result, businesses can focus their market intelligence strategies on a variety of goals, such as product launches and advertising; management of the brand and publicity;
fostering customer retention; delivering customer-specific services; keeping an eye on competitors and market trends;
reducing risk; reducing expenses and expanding the business as a whole.
A new and expanding field of study known as "sentiment analysis" is being fueled by the application of the big data phenomenon to social media. Its objective is to be aware of what people say and share on a daily basis. This data is used by businesses to better understand their customers and their operations. Educational institutions could also "listen" to students to learn more about how they see the world. Sentiment analysis is a useful tool for obtaining information about students' online behavior and, most importantly, their feedback on various aspects of the educational system, such as the university admissions process, features of qualifications, examinations, and their aspirations, by utilizing students' activity on social networking websites.
This information could be used by businesses to develop marketing strategies. This could be accomplished in a variety of ways, including focusing on countries or regions where students' online activity is lower than anticipated, monitoring their examination experiences based on discussions in online forums, comprehending the significance of their brand to students, and soliciting customer feedback on new products.

\subsection{Federated Learning}
Federated learning is a new idea that was just introduced by Google. To construct machine learning models, Google wants to use datasets that are dispersed
across various devices while limiting the leakage of
data. Recently, federated learning has seen advances
that address statistical issues [14,15] as well as security [12,16]. Personalization in federated learning is
also a focus of study [14,17]. Federated learning on
mobile devices involves dispersed mobile user interactions where communication costs in a large-scale
distribution, uneven data distribution, and device dependability are some of the important concerns for
optimization. In addition, data is partitioned horizontally in the data space by user IDs or device IDs. Data
privacy in a decentralized collaborative learning scenario is an important consideration in this area of research, which is why it has been linked to in [14]. We
broaden the original idea of “Federated Learning” to
include all privacy-preserving decentralized collaborative machine-learning approaches in order to address
collaborative learning situations within companies. An
introduction to federated learning and federated transfer
learning may be found in [18]. Security foundations
are further examined in this article, as are their connections to other fields of study, such as multi-agent theory
and privacy-preserving data mining. Federated learning
is a term that encompasses data partitioning, security,
and applications in this area. 
\subsubsection{Applications:}
Federated learning has a lot of promise as an innovative modeling mechanism [57,148,151] when used in the financial, sales, and other industries where data cannot be aggregated for the training of ML models due to factors like data security, privacy protection, and intellectual property rights [57,148,151]. Machine learning techniques will be used to provide customer-specific services like product recommendations and sales support. In the smart retail industry, some of the most important data elements include users' purchasing power, personal preferences, and product characteristics. In actual applications, there is a good chance that these three data elements will be distributed among three distinct departments or businesses.
\\
1. Wireless Communication
\\
Communication over wireless networks The complexity of wireless networks is increasing, so it is time to abandon outdated models-based approaches. In a similar vein, efforts to develop new models for wireless networks have shifted their focus in response to the growing popularity of deep learning networks [58,149]. Niknam et al. carried out an in-depth investigation of edge computing and 5G networks. 59] making use of the fundamental FL features. Simulations were carried out with standard data sets in order to demonstrate the security and dependability of FL in wireless communication.
\\
2. Service Recommendation 
\\
In 2016, it was made public that Google was working on a project to develop FL for Android mobile users [12]. The goal of this project was to improve the prediction accuracy of keyboard inputs while also safeguarding user privacy and security. The advancement of the language model will assist in the development of the recommendation system [66]. It can be used for a wide range of recommendation applications with federated learning. The model may provide an immediate response to any request made by a user.
\\

\section{Conclusion}
The concept of federated learning, which protects privacy across platforms, was discussed. More and more academics and businesses are recognizing federated learning as a viable paradigm when it comes to data privacy and security. When users are unable to train good models due to a lack of data, collaborative learning may combine the integrated model and update many user models without disclosing the source data. However, when users only have a limited amount of data, federated learning may also provide a secure method for sharing models and migrate models to specific tasks to address the issue of inadequate data labels.
I have discussed two current useful technologies in this paper: Big Data and Social Media, in addition to the primary issues and concerns they raise in these research fields.
This is an ever-increasing concern, and this research may help in slowing this thing down and building better models specifically for this purpose. We can develop model for node with the assistance of which we can protect privacy and information that is shared on social media.\vskip6pt

\enlargethispage{20pt}




%%%%%%%%%% Insert bibliography here %%%%%%%%%%%%%%

\vskip2pc



\begin{thebibliography}{9}

\bibitem{1}Big Data and Social Media Analytics, Bilkent yerleşkesi, Turkish Ministry of Health, Çankaya, Turkey
Mehmet Çakırtaş
Computer Engineering, Istanbul Medipol University, Istanbul, Turkey
Mehmet Kemal Ozdemir 

\bibitem{2} Impact of big data and social media on society
Savita Kumari 
(Sheoran) Assistant Professor Indira Gandhi University Meerpur, Rewari – 122502
 Cambridge, UK: UIT Cambridge.

\bibitem{3} Big data analytics meets social media: A systematic review of 
techniques, open issues, and future directions 
Sepideh Bazzaz Abkenar a
, Mostafa Haghi Kashani b
, Ebrahim Mahdipour a,*
, 
Seyed Mahdi Jameii b 
a Department of Computer Engineering, Science and Research Branch, Islamic Azad University, Tehran, Iran b Department of Computer Engineering, Shahr-e-Qods Branch, Islamic Azad University, Tehran, Iran 

\bibitem{4}International Journal of Hybrid Intelligent Systems 18 (2022) 19–35 19DOI 10.3233/HIS-220006IOS PressFederated learning: Applications, challengesand future directionsSubrato Bharatia,∗, M. Rubaiyat Hossain Mondala, Prajoy Podderaand V.B. Surya Prasathb,c,d,eaInstitute of Information and Communication Technology, Bangladesh University of Engineering and Technology,Dhaka, BangladeshbDivision of Biomedical Informatics, Cincinnati Children’s Hospital Medical Center, Cincinnati, OH, USAcDepartment of Pediatrics, University of Cincinnati College of Medicine, Cincinnati, OH, USAdDepartment of Biomedical Informatics, College of Medicine, University of Cincinnati, Cincinnati, OH, USAeDepartment of Electrical Engineering and Computer Science, University of Cincinnati, Cincinnati, OH, USA

\bibitem{5}Practical anonymization for data streams: z-anonymity and
relation with k-anonymity✩
Nikhil Jha a,∗
, Luca Vassio a
, Martino Trevisan b
, Emilio Leonardi a
, Marco Mellia a
a
Politecnico di Torino, Corso Duca degli Abruzzi, 24, Torino 10129, Italy
b Università degli Studi di Trieste, Dipartimento di Ingegneria e Architettura, Via Alfonso Valerio, 6/1, Trieste 34127, Italy

\end{thebibliography}

\end{document}
