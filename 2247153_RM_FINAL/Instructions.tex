\documentclass{rsproca_new}

\pagestyle{empty}
\begin{document}

\enlargethispage{24pt}

\section{Instructions for RS \LaTeX\ Template}

This latex class file is available for authors to prepare the manuscript for the Royal Society journals.
It is assumed that the authors are familiar with either plain \TeX, \LaTeX,\ \AmS-\TeX\ or a standard latex set-up,
hence only the essential points are described in this document. For more details please see the \textit{\LaTeX\ User's Guide}
or \textit{The not so short introduction to \LaTeXe}.

\section{Installation}
Within the supplied set of files, \texttt{rsproca\_new.cls} need to be copied into a directory
where tex looks for input files. The other files need to be kept as a reference while preparing your manuscript.
Please use pre-defined commands from \texttt{RSPA\_Author\_tex.tex} for title, authors, address, abstract, keywords, body etc.

\section{How to start using rsproca\_new.cls}
Before you type anything that actually appears in the paper you need to include a \verb+\documentclass{rsproca_new}+ command
at the very beginning and then, the two commands that have to be part of any latex document, \verb+\begin{document}+
at the start and the \verb+\end{document}+ at the end of your paper. The main structure of your document should be as follows:

\begin{verbatim}
\documentclass{rsproca_new}
\def\titlehead{Research}%%Please insert respective article type here
\begin{document}
\title{...}
\author{....}
\address{...}
\subject{...}
\keywords{...}
\corres{...}

\begin{abstract}
..........
..........
\end{abstract}

\begin{fmtext}
............
\end{fmtext}

\maketitle
....
\section{....}
...
\subsection{....}
....
\end{document}
\end{verbatim}

Note: In order to insert few lines on first page use the command \verb+\begin{fmtext}....\end{fmtext}+ before \verb+\maketitle+ command.
Please look into sample.tex for more clarification.

\section{Preamble}

In the preamble portion i.e. before \verb+\begin{document}+ insert the respective journal name:
\verb+\jname{rspa}+ and \verb+\Journal{Proc R Soc A\ }+ to get journal name in running heads.

\section{Packages used by the class file}

There are some packages that are essential when using the class file:
\begin{verbatim}
amsmath amssymb amsfonts amsthm graphicx endfloat endnotes setspace
verbatim geometry times helvet courier mathtime bm url babel dcolumn
\end{verbatim}

\noindent Some commonly-used packages are already used by this class file:
\begin{verbatim}
xspace amscd rotating latexsym multicol array algorithm subfigure
\end{verbatim}

\noindent Note:

For acknowledgement section use the command \verb+\ack{}+.


\vskip2.5pc

\noindent The rsproca\_new file is intended as a guide only. Article lengths based
on this estimation may be subject to change when the article is prepared for publication.
Authors whose article is estimated close to the page charge limit should contact the office regarding page charges.


\end{document}
